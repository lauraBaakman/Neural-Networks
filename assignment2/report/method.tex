%!TEX root = practicum2.tex
Given a dichotemy $\mathcal{D} = \left\{\xi^i, S^i \right\}_{i}^{N}$, $N$ $d$-dimensional patterns $\xi \in \mathcal{R}^d$, each with in contrast to the Rosenblatt algorithm where the labels were assigned randomly, a training label is defined by a teacher perceptron as:
\begin{equation}
	S^\mu = \text{sign}(\vec{w}^* \cdot {\vec{\xi}}^{\mu})
\end{equation}
The $\vec{w}^*$ can be chosen randomly or could be fixed, without loss of generality because the optimal solution is unique. \todo[inline]{because reasons... ofzo}.

The Minover algorithm runs over a period of time $t$ until either the stability of the solution does not change anymore or the number of maximal time steps $t_{max}$ is reached. The  stability is defined as: 
\begin{equation}\label{eq:method:maximum_stability}
\mathnormal{k}(t) = \frac{\vec{w}(t) \cdot \vec{\xi}^v S^v}{|\vec{w}(t)|}
\end{equation} \todo[inline]{v}
The formula in \eqref{eq:method:maximum_stability} can be thought of as the distance between all the patterns and the current solution $\vec{w}(t)$. To find the maximal stability, at every time step the algorithm selects the pattern with the smallest distance to the current solution. Then the formula in \eqref{eq:method:update} is used to update the $\vec{w(t + 1)}$.
\begin{equation}\label{eq:method:update}
	\vec{w}(t + 1) = \vec{w}(t) + \frac{1}{N} \xi^{\mu(t)} S^{\mu(t)} 
\end{equation}\todo[inline]{$\mu(t)$}

\begin{equation}\label{eq:method:generalization_error}
	\epsilon_g(t) = \frac{1}{\pi} arccos (\frac{\vec{w}(t) \cdot \vec{w}^*}{|\vec{w}(t)| |\vec{w}^*}|)
\end{equation}