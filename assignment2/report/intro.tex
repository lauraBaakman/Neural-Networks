%!TEX root = practicum2.tex
The first perceptron was designed by Rosenblatt in 1957 one of the variations inspired by this is called the Min-Over algorithm by Krauth and Mezard, 1987. \todo[inline]{Real reference}  The aim of the Minover algorithm is to find the largest margin between two separable classes to ensure that when using the perceptron during classification, the chance that the perceptron classifies the data point correctly is higher than when a non-optimal solution is used. The image in figure xa. illustrates the use of an optimal solution as apposed to using an non-optimal solution, shown in xb.

\todo[inline]{Plaatje xa. Optimal Minover solution vs Plaatje xb. Standard non-optimal solution }

\todo[inline]{Bla bla bla bla, bla bla bla bla. Bla bla bla.}