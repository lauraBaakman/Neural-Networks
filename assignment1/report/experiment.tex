%!TEX root = practicum1.tex
Since \autoref{eq:1:lsChance} only holds when the points are general position we sampled them from a Gaussian distribution for both experiments, since the points are then guaranteed to be linearly independent \cite[Chapter~5]{prince2012computer}. In these experiments we have made the number of patterns dependent upon some parameter $\alpha$ according to: 
	\begin{equation}\label{eq:alpha}
		N = \alpha d.
	\end{equation}
We define $Q_{l.s.}$ as the fraction of data sets where the perceptron found a hyperplane to separate the two classes within the $d_{max}$ epochs.	

\subsection*{Experiment I}\label{ssec:experimentI}
For the first experiment we have sampled 500 50-dimensional from a standard normal distribution. The number of data points per data set is dependent upon \eqref{eq:alpha}, $d_{max}$ was set to 1000. The ratio $Q_{l.s.}$ for different values of $\alpha$ is plotted in \autoref{fig:experiment1:plot}.

\begin{figure}[H]
	\centering
	\includegraphics[width=\columnwidth]{./img/Aa_N50_nd500_nmax1000}
	\caption{The parameter $\alpha$ versus $Q_{l.s.}$.}
	\label{fig:experiment1:plot}
\end{figure}

\Cref{fig:experiment1:plot} shows that while $\alpha = 1$ $Q_{l.s.}$ is also one, \todo{Why is this correct}. 

\todo[inline]{Resultaat experiment beschrijven}
\todo[inline]{Waarom wijkt het resultaat van het experiment af van de theorie}

\subsection*{Experiment II}
% Observe the behavior of Ql.s. for different system sizes N as well. Does it approach a step function with increasing N? To this end, repeat the above experiments for, say, N = 50 and N = 100.

In this experiment we repeat the previous experiment I (see \vref{ssec:experimentI}), with different sizes for $d$ in order to find if $Q_{l.s.}$ approaches a step function. The resulting plots for this experiment are shown in \vref{fig:2:experiment}. 

\begin{figure*}
	\centering
	\begin{subfigure}{0.48\textwidth}
		\centering
		\includegraphics[width=\textwidth]{./img/Ab_N50_nd75_nmax250}
		\caption{$d=50$}
		\label{fig:2:experiment:N50}
	\end{subfigure}
	\begin{subfigure}{0.48\textwidth}
		\centering
		\includegraphics[width=\textwidth]{./img/Ab_N75_nd75_nmax250}
		\caption{$d=75$}
		\label{fig:2:experiment:N75}
	\end{subfigure}

	\begin{subfigure}{0.48\textwidth}
		\centering
		\includegraphics[width=\textwidth]{./img/Ab_N100_nd75_nmax250}
		\caption{$d=100$}
		\label{fig:2:experiment:N100}
	\end{subfigure}	
	\begin{subfigure}{0.48\textwidth}
		\centering
		\includegraphics[width=\textwidth]{./img/Ab_N125_nd75_nmax250}
		\caption{$d=125$}
		\label{fig:2:experiment:N125}
	\end{subfigure}

	\caption{Number of successful runs $Q_{l.s.}$ depending on $\alpha = N / d$.}
	\label{fig:2:experiment}
\end{figure*}



\todo[inline]{Wat gaan we testen?}
\todo[inline]{Hoe gaan we het testen?}
\todo[inline]{Wat zijn onze onze resultaten}
\todo[inline]{In hoeverre kloppen de resultaten?}